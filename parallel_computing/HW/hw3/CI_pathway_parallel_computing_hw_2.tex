\documentclass[11pt]{article}
\usepackage{graphicx}
\usepackage{amssymb}
%\usepackage{booktabs}
\usepackage{array}
\usepackage[utf8]{inputenc}



\usepackage{epstopdf}
\usepackage{caption}


\usepackage[fleqn]{amsmath}



\usepackage{tipx}
\usepackage{tipa}

\usepackage{breakcites}
%\usepackage{/usr/local/texlive/2020/texmf-dist/tex/latex/breakcites/breakcites}

%\usepackage{supertabular}
%\usepackage{wasysym}

%\usepackage{setspace}

%\usepackage{pifont}



\usepackage{enumitem}

\usepackage{float} %%%% use H! position -- conserve vertical space

%\usepackage{mathabx}


%\usepackage{txfonts}


%\usepackage{Sweave}

\usepackage{fancyvrb} %%% for \VerbatimInput


%%%%%%%%%%%%%%%%%%%%%%%% FOR HYPER REF
\usepackage{xcolor}
\definecolor{MyDarkGreen}{rgb}{0.0,0.4,0.0}
\definecolor{MyDarkRed}{rgb}{0.4,0.0,0.0} 
\definecolor{MyBlue}{rgb}{0.0, 0.0, 0.5} 

\definecolor{MyOrange1}{rgb}{1.0, 0.9, 0.0} 

\usepackage[colorlinks=true, urlcolor= MyDarkRed, linkcolor= MyBlue, citecolor=MyDarkGreen ]{hyperref}

%\usepackage[colorlinks=false, urlcolor= MyOrange1, linkbordercolor=MyOrange1, citecolor=MyDarkGreen ]{hyperref}

\usepackage{makeidx}

\usepackage{listings}
\lstdefinestyle{dm4ds_lstCustom_01}{
  basicstyle=\ttfamily\footnotesize,
  breaklines=true
}


\definecolor{LLblue}{rgb}{0.153,0.455,0.682}
\definecolor{LLgold}{rgb}{1,0.82,0}
\definecolor{LLdarkestBlue}{rgb}{0,0.231,0.361}
\definecolor{LLdarkerBlue}{rgb}{0,0.333,0.529}
\definecolor{LLlighterBlue}{rgb}{0.545,0.722,0.91}
\definecolor{LLlightestBlue}{rgb}{0.855,0.922,0.996}
\definecolor{LLdarkestGold}{rgb}{1,0.722,0.11}
\definecolor{LLdarkerGold}{rgb}{1,0.78,0.173}
\definecolor{LLflagGreen}{HTML}{4C8944}
\definecolor{LLflagYellow}{rgb}{1,0.82,0}
\definecolor{LLflagRed}{HTML}{C41B2D}
\definecolor{LLdarkGray}{rgb}{0.2, 0.2, 0.2}
\definecolor{LLmediumGray}{rgb}{0.4, 0.4, 0.4}
\definecolor{LLdarkGreen}{rgb}{0.0, 0.3, 0.0}
\definecolor{LLemerald}{HTML}{50C878}


\lstdefinestyle{dm4ds_lstCustom_01}{
    language=R,
    basicstyle=\footnotesize\ttfamily\bfseries,
    %basicstyle=\footnotesize\courier,
    breaklines=true,
     alsoletter={_},
    %stringstyle=\color{LLdarkestBlue},
     %stringstyle=\color{LLdarkerGold},
     %stringstyle=\color{Emerald}\ttfamily
     stringstyle=\color{LLemerald}\ttfamily,
    %otherkeywords={0,1,2,3,4,5,6,7,8,9},
    otherkeywords={!,!=,~,$,\&,\%/\%,\%*\%,\%\%,<-,<<-},
    morekeywords={TRUE,FALSE, data.frame, as.data.frame, xtable, writeLines},
    %deletekeywords={data,frame,length,as,character},
    deletekeywords={include.rownames},
    keywordstyle=\color{LLblue}\bfseries,
    commentstyle=\color{LLmediumGray}
}





\usepackage{geometry}

\usepackage{textcomp}
\usepackage{multirow}
\usepackage{float}

%\usepackage[colorlinks=true, urlcolor= \rgb{0.0,0.2,0.0}  ]{hyperref}




%\singlespacing
%\onehalfspacing
%\doublespacing

%\pagestyle{empty} %% turn off page numbering

\DeclareCaptionLabelSeparator{space}

\DeclareGraphicsRule{.tif}{png}{.png}{`convert #1 `dirname #1`/`basename #1 .tif`.png}
\textwidth = 6.5 in
\textheight = 8.2 in
\oddsidemargin = 0.0 in
\evensidemargin = 0.0 in
\topmargin = 0.0 in
\headheight = 0.0 in
\headsep = 0.7 in
\parskip = 0.2in
\parindent = 0.0in
\newtheorem{theorem}{Theorem}
\newtheorem{corollary}[theorem]{Corollary}
\newtheorem{definition}{Definition}
%\title{Brief Article}
%\author{The Author}



\setcounter{secnumdepth}{3}
\setcounter{tocdepth}{3}

\makeindex

\begin{document}


%\newcommand\textcode\Verb


%\maketitle



\newif\ifuselocaldir
\uselocaldirtrue
\uselocaldirfalse


\newcommand{\DXZ}{
\begin{flushright}
\vspace{-.4in}
 { \raisebox{0.30ex}{{\tiny D}}\hspace{0.008in}X\hspace{0.01in}\raisebox{0.30ex}{{\tiny Z}}     }
\end{flushright}
}


\newenvironment{myQuote}[2]%
               {\begin{list}{}{\leftmargin#1\rightmargin#2}\item{}}%
               {\end{list}}

\begin{myQuote}{2cm}{2cm}
\begin{center}
{\huge
\textbf{CI Pathway: Parallel Computing} \\[0.4cm]
Assignment: Distributed Memory Parallelism
} \\[0.4cm]
\end{center}
\end{myQuote}

%\begin{myQuote}{3cm}{3cm}
%\begin{center}
%PRILIMINARY \& INCOMPLETE \\
%\end{center}
%\end{myQuote}



\begin{myQuote}{3cm}{3cm}
{\normalsize
\begin{center}
UCLA, Statistics\\Hochan Son\\Summer 2025\\
\today
%2012-03-12
\end{center}
}
\end{myQuote}


%\begin{abstract}
%{\normalsize
%Here's my abstract
%}
%\end{abstract}


%\newpage


%%\tableofcontents


%\newpage


%\input{_example.tex}

%\chapter{Introduction}

%Our textbook is \cite{briney2015data}.


%%%%%%%%%%%%%%%%




\section{Introduction}

\section{Hardware Environment}

\subsection{System Specifications}
The experiments were conducted on the NCSA Delta HPC cluster, which is a high-performance computing environment designed for parallel processing tasks. The specifications of the system are as follows:
\begin{table}[H]
\centering
\caption{NCSA Delta Compute Environment}
\label{tab:system_specs}
\begin{tabular}{@{}ll@{}}
\hline
\textbf{Component} & \textbf{Specification} \\
\hline
Compute Platform & NCSA Delta HPC Cluster \\
Login Node & dt-login04.delta.ncsa.illinois.edu \\
Compute Node & cn094.delta.ncsa.illinois.edu \\
Operating System & Linux 4.18.0-477.95.1.el8\_8.x86\_64 \\
Distribution & Red Hat Enterprise Linux 8 \\
Architecture & x86\_64 \\
Node Interconnect & HPE Slingshot \\
\hline
\end{tabular}
\end{table}


\subsection{Processor Architecture}

\begin{table}[H]
\centering
\caption{AMD EPYC 7763 Processor Specifications}
\label{tab:processor_specs}
\begin{tabular}{@{}ll@{}}
\hline
\textbf{Parameter} & \textbf{Value} \\
\hline
CPU Model & AMD EPYC 7763 64-Core Processor \\
Architecture & AMD Zen 3 (Milan) \\
Physical Cores & 64 per socket \\
Hardware Threads & 128 (2-way SMT) \\
Base Clock & 2.45 GHz \\
Boost Clock & Up to 3.5 GHz \\
Manufacturing Process & 7nm TSMC \\
Socket Type & SP3 \\
\hline
\end{tabular}
\end{table}



\section{Exercises For This Module.}
Write a code that runs on 8 PEs and does a "circular shift." This means that every PE sends some data to its nearest neighbor either "up" (one PE higher) or "down." To make it circular, PE 7 and PE 0 are treated as neighbors. Make sure that whatever data you send is received. The easiest way to do that is to have each PE print out its received messages.

%\newgeometry{top=1cm, left=1cm, right=1cm}
%\begin{figure}[htbp]
%\begin{center}
%\includegraphics[width=17.5cm]{_assets/bigStudentPicArray_01.jpeg}
%\caption{UCLA Graduate Statistics Students}
%\label{fig:studentArray}
%\end{center}
%\end{figure}
%\restoregeometry


\subsection{Exercise Notes}

\begin{enumerate}
  \item {To compile with MPI we do either:

  \begin{verbatim}
    mpicc laplace_mpi.c
    
    or
    
    mpif90 laplace_mpi.f90
  \end{verbatim}
  \\You will have an executable called a.out. Now you need to ask for a compute node with 8 processes allocated in order to run. Similar, but not identical, to our previous Slurm command:
  \begin{verbatim}
    srun --account=becs-delta-cpu --partition=cpu-interactive \
      --nodes=1 --tasks=8 --tasks-per-node=8 --pty bash
  \end{verbatim}
  \\And we wish to run using 8 processes. The command to run our a.out executable or available process is
  \begin{verbatim}
    mpirun -n 8 a.out
  \end{verbatim}
  Submit a copy of your code. Any output might be helpful, too.
  }

\section{Solution Implementation}


\section*{Conclusion}


%\newpage{}
%\pagebreak{}

\bibliographystyle{plain}

\bibliography{Lab_X} 


\newpage

\section{Appendix.code}

Here's some of our code (Note the use of VerbatimInput from package \texttt{fancyvrb}):


\subsection{Code A: prime\_serial.c}

%\begin{footnotesize}
%\VerbatimInput{_code_A.R}
%\end{footnotesize}

\begin{footnotesize}
\lstinputlisting[style=dm4ds_lstCustom_01]{./prime_serial.c}
\end{footnotesize}




\subsection{Code B: prime\_parallel\_race.c}

%\begin{footnotesize}
%\VerbatimInput{_code_B.R}
%\end{footnotesize}

\begin{footnotesize}
\lstinputlisting[style=dm4ds_lstCustom_01]{./prime_parallel_race.c}
\end{footnotesize}



\subsection{Code C: prime\_parallel\_norace\_1.c}

%\begin{footnotesize}
%\VerbatimInput{_code_B.R}
%\end{footnotesize}

\begin{footnotesize}
\lstinputlisting[style=dm4ds_lstCustom_01]{./prime_parallel_race.c}
\end{footnotesize}



\subsection{Code C: prime\_parallel\_norace\_2.c}

%\begin{footnotesize}
%\VerbatimInput{_code_B.R}
%\end{footnotesize}

\begin{footnotesize}
\lstinputlisting[style=dm4ds_lstCustom_01]{./prime_parallel_race.c}
\end{footnotesize}



% \subsection{Code C: laplace\_omp\_parallel.c}

% %\begin{footnotesize}
% %\VerbatimInput{_code_C.R}
% %\end{footnotesize}

% \begin{footnotesize}
% \lstinputlisting[style=dm4ds_lstCustom_01]{ex1/laplace_omp_parallel.c}
% \end{footnotesize}


% \subsection{Code D: laplace\_mpi.c}

% %\begin{footnotesize}
% %\VerbatimInput{_code_C.R}
% %\end{footnotesize}

% \begin{footnotesize}
% \lstinputlisting[style=dm4ds_lstCustom_01]{ex2/laplace_mpi.c}
% \end{footnotesize}



\section{Appendix.hw2\_result.txt}

\subsection{result.txt}
\begin{footnotesize}
 \lstinputlisting[style=dm4ds_lstCustom_01]{./hw2_result.txt}
\end{footnotesize}

% \subsection{ex2\_result.txt}
% \begin{footnotesize}
%  \lstinputlisting[style=dm4ds_lstCustom_01]{ex2/ex2_result.txt}
% \end{footnotesize}


\end{document}